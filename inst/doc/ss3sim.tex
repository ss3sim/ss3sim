\documentclass[12pt]{article}
\usepackage{geometry} 
\geometry{letterpaper}
\usepackage{graphicx}
\usepackage{Sweave}

% \VignetteIndexEntry{How to use the ss3sim package to run simulations in SS3}
% \VignetteKeyword{metapopulations}
% \VignetteKeyword{ecology}

%\usepackage[round]{natbib} 
%\bibliographystyle{apalike}   
%\bibpunct{(}{)}{;}{a}{}{;}   

\title{How to use the \texttt{ss3sim} package to run\\simulations in SS3}
%\author{Sean C. Anderson and others to be included}
\date{}

\begin{document}
\maketitle

\noindent
First start by installing the latest version of \texttt{ss3sim} and loading the
package:

\begin{Schunk}
\begin{Sinput}
> # install.packages("devtools")
> # devtools::install_github("ss3sim", username="seananderson")
> library(ss3sim)
\end{Sinput}
\end{Schunk}

\section*{Setting up the file structure}
We are assuming there are a series of operating models in a folder and a series
of estimation models in another folder. Within each folder, the models should be
named according to whatever you would like the scenario ID to be. For our
purposes, I suggest we use a brief identifier made up of lower-case letters and
numbers followed by a dash followed by the species name. For example for a
scenario with a block change in natural mortality you might have these folders:

\begin{verbatim}
blockm-cod
blockm-flat
blockm-sardine
\end{verbatim}

\noindent
It is up to the various groups to come up with these operating models and
estimation models. There are a number of functions in this R package to
facilitate this. We will come back to this. 

Once you have these folders set up you can move them into the simulation folder
structure with the \texttt{copy\_model} function. Assuming you've put these in
folders called \texttt{operating-models} and \texttt{estimation-models} you can
copy the models over like this:

\begin{Schunk}
\begin{Sinput}
> copy_models(model_dir = "operating-models", type = "om")
> copy_models(model_dir = "estimation-models", type = "em")
\end{Sinput}
\end{Schunk}

\noindent
or if you were only responsible for 1:50:

\begin{Schunk}
\begin{Sinput}
> copy_models(model_dir = "operating-models", type = "om", 
+   iterations = 1:50)
\end{Sinput}
\end{Schunk}

\noindent
This creates the structure:

\begin{verbatim}
  blockm-cod/1/om
  blockm-cod/1/em
  blockm-cod/2/om
  blockm-cod/2/em
  ...
\end{verbatim}

\noindent
Note that the operating and estimating model folders have been renamed
\texttt{om} and \texttt{em} within each iteration.

The functions in this package assume you've set your working directory
in R to be the base folder where you will store the scenario folders. The
folders containing the operating and assessment scenarios should also be in this
same base folder.

\section*{Running the models}

The \texttt{run\_scenario} function is a wrapper function. It calls
\texttt{run\_model} to run the operating model, adds the recruitment deviations,
samples various survey estimates from the operating model, copies and renames
files as necessary, and calls \texttt{run\_model} again to run the estimation
model.

Say you have a text files of scenarios to run and you want to run the first 50
iterations of those scenarios. You could run them like this:

\begin{Schunk}
\begin{Sinput}
> scenarios <- scan("mysenarios.txt", what = "character")
> run_scenario(scenarios, iterations = 1:50)
\end{Sinput}
\end{Schunk}

\noindent
Or, to test the operating model for the first scenario only:

\begin{Schunk}
\begin{Sinput}
> run_scenario(scenarios[1], iterations = 1, type = "om")
\end{Sinput}
\end{Schunk}

\section*{The flat scenario ID structure}
There are many advantages to this flat scenario ID fold setup:

\begin{enumerate}
  \item It makes it easier for multiple papers to share scenarios.

  \item It makes it easier for papers to change which scenarios to compare after.

  \item It avoids unnecessary nested folder structure.

  \item It's easier to distribute the model runs across people and computers.

  \item The functions are more general and applicable to future research.

  \item Since each folder represents a unique scenario run, it's simple to keep
    track of progress on model runs in a spreadsheet
  
\end{enumerate}

\noindent
Cole suggested we have a spreadsheet with the following columns:

\begin{verbatim}
Scenario ID, Scenario description, Control modifications, Model status
\end{verbatim}

\noindent
Then, groups can compile a list of scenario IDs they want to extract and compare.

\section*{Setting up the models}

The main functions to work with are:

\begin{description}
  \item[\texttt{change\_f}] A function to alter fishing mortality values in the
    \texttt{.par} file

  \item[\texttt{add\_time\_varying\_features}] Adds time-varying natural
    mortality either through the environment, block, or deviation methods.
\end{description}

ADD EXAMPLES OF USING THESE FUNCTIONS FOR COMMON SCENARIOS

\section*{What \texttt{run\_scenario} does}

Between running the operating and estimation models, \texttt{run\_scenario}
performs a number of tasks that are needed across all scenarios:

\begin{enumerate}
  \item Takes the appropriate column of recruitment deviations from
    \texttt{data(recdevs)}, scales them by the appropriate standard deviation
    using \texttt{TODO}, and adds them to the \texttt{.par} file using
    \texttt{change\_rec\_devs}.
  \item Samples the length and age composition data and adds them to the data
    file using \texttt{change\_lcomp} and \texttt{change\_agecomp}.
  \item Jitters the index of abundance based on the reported biomass for each
    fleet using \texttt{jitter\_index}.
  \item TODO renames and moves files as needed
\end{enumerate}

%\noindent
%These functions are used internally by \texttt{run\_simulations} between running
%the operating and estimation models:

%\begin{description}
  %\item[\texttt{change\_lcomp}] Takes a \texttt{data.SS\_new} file, resamples
    %the length compositions from the expected values, and returns a new file
    %with the new length comp samples. Samples can have dimensions, bins, sample
    %sizes, and distributions which are different than those coming from SS.

  %\item[\texttt{change\_agecomp}] Similar to \texttt{change\_lcomp}
    %but for age composition data.

  %\item[\texttt{jitter\_index}] This function is used to create an index of
    %abundance sampled from the expected available biomass for each fleet: survey
    %1 and survey 2 (which mimics the fishery) and add some lognormal errors
    %around it. 
%\end{description}



%\bibliography{/Users/seananderson/Dropbox/tex/jshort,/Users/seananderson/Dropbox/tex/ref}
\end{document}


